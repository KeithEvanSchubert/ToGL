\chapter{Why I Don't Celebrate Halloween}\label{a-nohalloween}

Halloween is the second largest holiday in the United States by the metric of money spent in its celebration.  Many people find it strange that I don't want to celebrate something that is so much a part of the national ethos, so I have written this as a brief explanation of why.

Quite simply, I don't celebrate Halloween because it is the holiday of a religion that is not mine.  I don't expect Jewish people to celebrate Christmas, or Wiccans to celebrate Pentecost.  It would not make sense.  Why do so many expect me to celebrate the holiday of a non-Christian religion?  To some it will seem strange that Halloween is a religious holiday, because they think of Halloween as merely a time to dress up and get candy.  I will spend the next section covering the origin of Halloween in the Celtic festival of Samhain, not as a commentary on Celts, but to establish the origin and hence my desire not to celebrate it.

\section{History of Halloween}

While there are many mysteries in history and the origin of holidays, a few facts stand out about Halloween.  Halloween's primary origins are in Celtic tradition and most notably the festival of Samhain (pronounced Sow-in, with sow said like cow).  Samhain literally means `end of summer' and is not a Celtic god as some claim\footnote{The point is irrelevant anyway as the problem to Christians is that it is a Celtic religious festival.}.  The Celtic year ended in fall, with some disagreement as to what date in the range October 21-Nov 11 it actually occurred on.  Most people today claim it was Nov 1, so the last day of the year was October 31, though this is more due to the present date of Halloween than to any real evidence I have seen.  I personally believe the real date lay on the autumnal equinox as transition times like the equinoxes were considered particularly special to the Celts, but the actual date is irrelevant.

Regardless, Samhain was the end of year celebration and had elements of harvest and death.  This might seem an odd combo, but the Celt's felt that transitions were the most special times --- dawn and dusk were the special times of day and the vernal and autumnal equinoxes were high holy days.  During Samhain the fairy world was thought closer and so the dead were believed able to return and it was supposed to be easier to tell the future.  The Druids held a great feast, which involved at least animal sacrifices and some historical evidence indicates human sacrifices as well\footnote{I mention this not as a commentary on their religion, which I do feel is false, but rather as evidence on the religious significance of it.}.  Fires in houses were extinguished to allow them to be rekindled by fire from the Druidic sacred bonfire, which was central in the celebration\footnote{Some claim this was also to make the home not as hospitable for spirits.}.  Some sources indicate that there was a fear of possession at this time even going as far as saying that a person believed to be possessed was sacrificed at the ceremony as a warning to the spirits, but I have not been able to confirm this in a reputable source.  Whether a belief in possession was part of the Druidic holy day or not, it is easy to see that Samhain was a Druidic religious celebration.  A few well established characteristics of Samhain that continue today will illustrate the strong bond.
\begin{enumerate}
\item Superstitions around wearing disguises to be mistaken as other spirits or fairy folk are the progenitors of modern costume wearing.
\item Food was left out for spirits as appeasement, and door to door collections of food for the feast were taken\footnote{Some claim by children but I cannot verify this.}.  This is one root of the modern collection of candy.
\item Stories and images of the dead, apparitions and such were common.  One only needs to look at any Halloween decorations, haunted houses, books, and movies to see how this plays out today.
\end{enumerate}

In Celtic lands conquered by the Romans, some Roman feasts were integrated such as the goddess Pomona's symbol of apples (seen most notably in bobbing for apples).  The Roman Catholic church attempted to convert Samhain into one of its holidays by moving All Saints Day to November 1, but this only gave rise to the modern name of the holiday, as Halloween is derived from the Middle English name of the evening before All Saints Day.  The Roman Catholic holiday included `souling', where the poor could beg for a special bread with courant in exchange for prayers on behalf of the giver's family to get them out of purgatory.  The practice of souling is the other root of trick or treating, and as a protestant, I am opposed to both purgatory and the ability for any but God to effect someone's place after death.

Carving pumpkins is also of Celtic origin, but is not related to Samhain.  It relates to an Irish fairy tale of an evil man named Jack, who amongst other things, tricks the devil and ends up being denied entrance in heaven or hell.  He roams the earth lighting his way with a single coal in a carved turnip, and was referred to a Jack of the lantern, hence the modern Jack-o-lantern.  Pumpkins were discovered in the new world and have replaced turnips as the carved vegetable of choice.  The jack-o-lantern became part of Halloween celebrations because of characteristic 3 above, namely stories of the dead.  Again, not something I want to teach my children to do.

In short, pretty much everything done on Halloween are celebrations and extensions of Samhain, a Celtic religious holy day, with a little Roman Paganism and Roman Catholicism for good measure.

\section{Modern Practice}

I am not opposed to imagination, fantasy, or childhood fun, but I do oppose uplifting evil.  Think for a moment at what people dress their kids up as, and ask yourself if that is really what you want to teach your children it is fun to be.
\begin{itemize}
\item devils and demons - Really this should end it for Christians.  How can you partake of any celebration which exalts such a direct opposition to Christ and the Bible.  Some even dress themselves or their kids in such blatantly evil outfits.  Some might think me to be over-reacting, but honestly can you see wearing such things in Heaven before God?  Is that what we want to be a part of?
\item witches, warlocks, and cult members - Witchcraft is a religion that is growing in the US.  It has been becoming more and more popular in TV and movies, but it's first and still prime advertising to kids is through Halloween.  Modern witches think of Halloween/Samhain as their holiday, and this is generally accepted by most in the US.  Historically they are wrong, as modern witchcraft was made up in the last fifty years and Halloween is of Celtic and Roman origins, but Paul tells us in I Thessalonians 5:22 to abstain from even the appearance of evil.  Halloween is openly advertised as a celebration of terror, death, and evil.  Christian parents, who would be hurt, shocked, and appalled if their kids followed witchcraft think nothing of dressing them as witches at Halloween.
\item Zombies, vampires, ghosts, and dead things - Most of these are evil monsters who kill and eat people, not to mention it glorifies death, which we don't want to do.  Many parents are afraid to speak to kids about death and don't want them to go to relatives funerals, but they will gladly dress them up as a dead thing and say to have fun.  How warped is that?
\item Murderers, thugs, thieves, gangsters, and pirates - a huge number of costumes for kids and adults glorify those who kill, rape, steal, and in general harm others for their own benefits.  Most kids know far more about pirates than the apostles and we wonder why the nation is going the way it is.
\end{itemize}
Some may protest that they are dressing their kids as princesses and knights, and other happy things.  Remember that Jesus said a corrupt tree cannot bring forth good fruit (Matthew 7:17-18, Luke 6:43).  While Jesus is speaking directly about people who are not saved being able to bring forth good, the general principal is still valid here.  Halloween, as a pagan holiday which is very dark even in its modern version, cannot bring good to your child.  Further, Paul reminds us that evil company (communication) corrupts good manners.  If you put your kids in an evil practice, though you dress them nicely, they will be effected.  How can they not?  Would you be ok with you kid doing what criminals do, as long as they are dressed nicer?  Parents kid themselves because it is hard to say no to their kids.  It is hard to be the odd one.  God will support those who walk with Him, as he promises in I Corinthians 10, which we deal with in the next section.

Beyond the costumes, add the haunted houses, graveyards, and the like.  Some may object that haunted houses are fun.  I never said they weren't.  People wouldn't go if they weren't fun.  Life is not about the world's fun to a Christian, life is about God and having fun in worshiping Him.  As mentioned above about costumes, haunted houses also exalt evil, terror, and death.  That people find such things fun is a condemnation on humanity, not a justification.

Think seriously on ``trick or treat'' while everyone accepts and even thinks of it as fun, it is on the surface a threat - give me candy or I will do a trick to you.  This is actually how it started being practiced in the United States in the 1930's.  Papers all across the US at the time reported threats of pranks like soaping windows being threatened if treats were not given to the ``masked wearing ghosts and ghoulies.''  We think of it as innocent fun because it is accepted, but that was not the case 70 years ago, when many shot at kids for the threats.  That is definitely over-reacting, but honestly when else is it ok to demand candy from everyone in your neighborhood?  As a side note, even advertisements in the 1930's for candy had a witch on them.  Witches have been associated with the modern holiday from the start, even years before the modern religion started.

We could go on, but why?  An honest appraisal of the modern practice of Halloween, finds it in perfect agreement with the ancient.  While people may try to claim ignorance, the in-your-face nature of the evil celebrated in Halloween should end all support of it from Christians.

\section{God's view of Participating in Other Worship}

There are a great many places in the Bible that speak of the dangers and errors of participating in pagan worship.  For the sake of space I will use only I Corinthians 8-10, which has a lengthy discussion about meat sacrificed to idols.  I covered this briefly in the chapter on celebrating Resurrection Sunday rather than Easter, but it is worth going over in more detail.  The passage might seem irrelevant to us, but it is not.  Like it or not, Halloween/Samhain is a religious holiday of a pagan religion.  The modernization, sanitization, and lack of understanding by many involved is not relevant (I will deal with more of this later); the fact remain the celebration is pagan.  Before we get into the text, let us consider the background.

One question many first century Christians had was how to handle non-Christian ceremonies and the things involved in them.  Large celebration meals revolved around the pagan sacrifices, and portions of the sacrifice were usually taken home by the one bringing the sacrifice and the officiant, then the excess was sold in the shambles (meat market).  This created a big question for Christians.  Should they participate when invited because they knew the other ``gods'' really didn't exist?  Could they buy the cheap meat, or did they need to find out where it came from?  What about when they went to a friend's house, should they inquire the origin of the meal?  The Corinthians asked Paul for what they should do.  In I Corinthians 8-10 Paul answers their question.  We need to ask the same questions they did.  Let's look through these chapters.

\begin{quote}
\textsuperscript{1}  Now as touching things offered unto idols, we know that we all have knowledge. Knowledge puffeth up, but charity edifieth.
\textsuperscript{2}  And if any man think that he knoweth any thing, he knoweth nothing yet as he ought to know.
\textsuperscript{3}  But if any man love God, the same is known of him.
\begin{flushright}I Corinthians 8:1-3\end{flushright}\end{quote}

Paul starts by reminding those who disagreed that everyone has knowledge, but we are not to rely solely on our knowledge we are to have love (charity) for one-another as the goal is to edify (build up in Christ) each other. He also reminds us that none of us knows everything so we should all be humble because Christians are all still growing in Christ.  The important thing is to love God and put Him first.  This is a good point for all of us to remember, and I don't exclude myself from it. I go to God in prayer and study of His word to seek His correction and modification of my thinking, and so should we all.  Let us proceed by exalting Christ and not our own will and thoughts.

\begin{quote}
\textsuperscript{4}  As concerning therefore the eating of those things that are offered in sacrifice unto idols, we know that an idol is nothing in the world, and that there is none other God but one.
\textsuperscript{5}  For though there be that are called gods, whether in heaven or in earth, (as there be gods many, and lords many,)
\textsuperscript{6}  But to us there is but one God, the Father, of whom are all things, and we in him; and one Lord Jesus Christ, by whom are all things, and we by him.
\begin{flushright}I Corinthians 8:4-6\end{flushright}\end{quote}

Paul then notes that the other religions are fake.  We have nothing to fear from false religions and their practices.  I don't fear ghosts, witches, or any other such things.  They are false and my God reigns.  We don't avoid something because of fear of others practice.  Many Christians fear the world or its ability to get their children.  That is silly, nothing is able to get past God.  He is sovereign over everything, which is a point He says over and over again in the Bible.

\begin{quote}
\textsuperscript{7}  Howbeit there is not in every man that knowledge: for some with conscience of the idol unto this hour eat it as a thing offered unto an idol; and their conscience being weak is defiled.
\textsuperscript{8}  But meat commendeth us not to God: for neither, if we eat, are we the better; neither, if we eat not, are we the worse.
\textsuperscript{9}  But take heed lest by any means this liberty of yours become a stumblingblock to them that are weak.
\textsuperscript{10}  For if any man see thee which hast knowledge sit at meat in the idol�s temple, shall not the conscience of him which is weak be emboldened to eat those things which are offered to idols;
\textsuperscript{11}  And through thy knowledge shall the weak brother perish, for whom Christ died?
\textsuperscript{12}  But when ye sin so against the brethren, and wound their weak conscience, ye sin against Christ.
\textsuperscript{13}  Wherefore, if meat make my brother to offend, I will eat no flesh while the world standeth, lest I make my brother to offend.
\begin{flushright}I Corinthians 8:7-13\end{flushright}\end{quote}

Next he tells us that food is not what makes us right with God, so the food we eat or obstain from will not make us more or less spiritual.  The idol cannot make good food bad.  Note Paul is answering one big question: does the food become defiled by where it comes from or what is used in.  The answer is no, it is not defiled.  This does not say you can participate in the ceremony, only that the food is not made bad.  Putting this in the current situation, we shouldn't oppose candy, even at this time of year.  It is fine for kids to eat a little candy, even if it came from someone practicing Halloween.  It does not tell us we can participate in Halloween, only that the food isn't tainted.

Paul adds a caveat about eating the food, that must be observed.  Putting this in modern terms, he tells us that if someone offers us a snickers, and we go for it because it is tasty and we can, but a Christian friend says, ``Hey that was Halloween candy!''  I should say, ``Oh, no thanks then.''  I have the right to eat it, but it would hurt one for whom Christ died, and thus without bitterness and remorse I must abstain and be glad I could make brother or sister at ease.  Note it does not open up participation, it actually restricts our rights to do something if it will offend a brother or sister.  Note in particular verse 10, where Paul gives a hypothetical about sitting in an idol's temple.  Paul does not comment on the correctness of going to the temple in and of itself, he uses the extreme case of participation and comments on what effect it would have on a Christian brother or sister.  I Corinthians 10:15-22 specifically says you cannot eat at the idol's temple, here Paul is concerned with pointing out that our behavior has an effect on other believers.  I can lead a brother or sister into problems by either doing what is permitted or by committing sin.  Note that even in chapter 8's hypothetical, it refers to those with ``knowledge'' of being free as the one sitting in the temple, but Paul had already explained just a few verses ago that their ``knowledge'' puffs up and is no knowledge as it out to be known.  Their ``knowledge'' was hurting a weaker brother or sister, and thus even in attitude was sinful.  Paul never says it is ok, he shows even in attitude it does not reflect Christ.  One final point is important: this does not restrict our freedom for non-believers, and it does not restrict God's commands.  We can't say that we won't do something God commanded because someone is offended.  God's command is more important than man's preferences.

Paul uses chapter 9 to address attitude and freedom in his own life.  For length considerations I will just summarize the point that relates to us, and encourage you to read it yourself.  Paul notes he has freedom to do good things like have a wife and to be supported by the church, but that he forbear from these things so he could advance God's word and be easier on those he ministered to.  He uses his example of giving up good things for God, the gospel, and to build up his fellow believers in Christ as a contrast to committing a sin and causing a brother to be built up in sin.

\begin{quote}
\textsuperscript{1}  Moreover, brethren, I would not that ye should be ignorant, how that all our fathers were under the cloud, and all passed through the sea;
\textsuperscript{2}  And were all baptized unto Moses in the cloud and in the sea;
\textsuperscript{3}  And did all eat the same spiritual meat;
\textsuperscript{4}  And did all drink the same spiritual drink: for they drank of that spiritual Rock that followed them: and that Rock was Christ.
\textsuperscript{5}  But with many of them God was not well pleased: for they were overthrown in the wilderness.
\begin{flushright}I Corinthians 10:1-5\end{flushright}\end{quote}

Paul then picks up with an example from the Old Testament.  In the Exodus, the Jews all participated in what God did for them, which was to lead them to Christ the rock.  Even with all the participation, they were not all true believers, which can be seen in that God was not pleased with them and they wanted things other than God.  They wanted Egypt more.  They worshiped the golden calf.  Aaron's son's offered strange fire, because they wanted to worship their way and God slew them for it.  They complained it was hard.  God's way was no fun to them.  They wanted things there way.  So do we.  It is not an accident that it is here in the middle of a section on not participating in pagan worship.  People now defend Halloween not in it being God approved, but rather that it is fun for their kids.  Jesus said in Matthew 10:37-38, ``He that loveth father or mother more than me is not worthy of me: and he that loveth son or daughter more than me is not worthy of me. And he that taketh not his cross, and followeth after me, is not worthy of me.''  That is probably pretty hard for many to hear, but that is what God says and we are to follow.  Note He does not say we should not love our kids, you actually can't love them if you don't love God infinitely more than them.  To do the right thing you must do it God's way.

\begin{quote}
\textsuperscript{6}  Now these things were our examples, to the intent we should not lust after evil things, as they also lusted.
\textsuperscript{7}  Neither be ye idolaters, as were some of them; as it is written, The people sat down to eat and drink, and rose up to play.
\textsuperscript{8}  Neither let us commit fornication, as some of them committed, and fell in one day three and twenty thousand.
\textsuperscript{9}  Neither let us tempt Christ, as some of them also tempted, and were destroyed of serpents.
\textsuperscript{10}  Neither murmur ye, as some of them also murmured, and were destroyed of the destroyer.
\textsuperscript{11}  Now all these things happened unto them for ensamples: and they are written for our admonition, upon whom the ends of the world are come.
\textsuperscript{12}  Wherefore let him that thinketh he standeth take heed lest he fall.
\textsuperscript{13}  There hath no temptation taken you but such as is common to man: but God is faithful, who will not suffer you to be tempted above that ye are able; but will with the temptation also make a way to escape, that ye may be able to bear it.
\textsuperscript{14}  Wherefore, my dearly beloved, flee from idolatry.
\begin{flushright}I Corinthians 10:6-14\end{flushright}\end{quote}

Paul now makes the connection explicit.  They were an example in that they lusted after evil, and in they same way Paul says believers should not participate in idolatry as they did in the desert.  Paul then mentions about the fornication committed by the Israelites with the Moabite women, which resulted in God sending serpents in to kill thousands of them.  Paul is reminding us of God's tremendous hatred of idolatry.  He follows this with a statement of not tempting Christ with our idolatry, or complaining (murmuring) because God does not accept that either.  Then we are told it was an example for us, and we are foolish to ignore it.  Many, who think themselves to be strong Christians, sadly tolerate sins such as pagan influences, thinking themselves unharmed by it.  Paul's warning should make us all examine our lives again.  The verse is famous, but its context is not, and so it is often misused.  It is warning us not to think ourselves so strong that we can ``handle'' have parts of our lives not given to Christ.  The next verse is also famous but its context forgotten, for the temptation to do the things of this world and fit in is great.  God is faithful, even when we are not what we should be, and He will provide us a way out.  First century Christians faced death for not participating in some pagan worship (like the emperor worship required by the Roman government).  We face the loss of dressing up like evil things and getting candy.  Are we really that shallow that this is a hard choice.  God says he will help.  Paul then pleads for us to flee idolatry.  Halloween is idolatry, for it is a modern, sanitized form of Celtic and Roman paganism.

\begin{quote}
\textsuperscript{15}  I speak as to wise men; judge ye what I say.
\textsuperscript{16}  The cup of blessing which we bless, is it not the communion of the blood of Christ? The bread which we break, is it not the communion of the body of Christ?
\textsuperscript{17}  For we being many are one bread, and one body: for we are all partakers of that one bread.
\textsuperscript{18}  Behold Israel after the flesh: are not they which eat of the sacrifices partakers of the altar?
\textsuperscript{19}  What say I then? that the idol is any thing, or that which is offered in sacrifice to idols is any thing?
\textsuperscript{20}  But I say, that the things which the Gentiles sacrifice, they sacrifice to devils, and not to God: and I would not that ye should have fellowship with devils.
\textsuperscript{21}  Ye cannot drink the cup of the Lord, and the cup of devils: ye cannot be partakers of the Lord�s table, and of the table of devils.
\textsuperscript{22}  Do we provoke the Lord to jealousy? are we stronger than he?
\begin{flushright}I Corinthians 10:15-22\end{flushright}\end{quote}

Paul now gets to brass tacks.  He knows they don't want to listen, just as I am aware most people who read my words will not like them, and will be angry at me.  Give Paul's words some thought.  He then cites the significance of the Lord's Table (or communion).  Paul notes that partaking of it is to commune with Christ.  He then notes it was the same in Old Testament Israel and the sacrifices.  What is the point?  He asks that same question in verse 19.  Paul's point is that pagan's are making their sacrifices to demons, whether they know it or not, and if we partake of the ceremony we are partaking of the demons just as the Lord's Table is partaking of Christ.  What does this have to do with a harmless kids holiday.  Nothing.  Halloween is not a harmless kids holiday.  Be honest, kids dress as demons, witches, lost souls/undead, criminals and so forth.  How can I say that it is wrong to participate in Halloween and that Halloween is demonic?  Simply because I am stating what is so obvious to anyone who is not trying to justify their own participation.  The influence is everywhere and in bold letters, people deny it because they don't want to give it up, just like alcoholics deny they need to drink.  Paul says you can't do both.  Paul says Jesus is jealous and will not tolerate us playing the harlot with the world as Israel did, and was so often punished.  Paul is emphatic that we should not participate in ceremony itself.  Do we really think we can take God?  That is exactly what Paul says, but still we do not listen.  Even churches put on Halloween festivals and ``Trunk or Treating''.  How insulting to God!  Near where I live they have even started doing their version on October 30 so people can do both.  The truth really shows in that.  It is not an alternative, it is another program to make people want to come, regardless of its spiritual bankruptcy.

\begin{quote}
\textsuperscript{23}  All things are lawful for me, but all things are not expedient: all things are lawful for me, but all things edify not.
\textsuperscript{24}  Let no man seek his own, but every man another�s wealth.
\textsuperscript{25}  Whatsoever is sold in the shambles, that eat, asking no question for conscience sake:
\textsuperscript{26}  For the earth is the Lord�s, and the fulness thereof.
\textsuperscript{27}  If any of them that believe not bid you to a feast, and ye be disposed to go; whatsoever is set before you, eat, asking no question for conscience sake.
\textsuperscript{28}  But if any man say unto you, This is offered in sacrifice unto idols, eat not for his sake that shewed it, and for conscience sake: for the earth is the Lord�s, and the fulness thereof:
\textsuperscript{29}  Conscience, I say, not thine own, but of the other: for why is my liberty judged of another man�s conscience?
\textsuperscript{30}  For if I by grace be a partaker, why am I evil spoken of for that for which I give thanks?
\textsuperscript{31}  Whether therefore ye eat, or drink, or whatsoever ye do, do all to the glory of God.
\textsuperscript{32}  Give none offence, neither to the Jews, nor to the Gentiles, nor to the church of God:
\textsuperscript{33}  Even as I please all men in all things, not seeking mine own profit, but the profit of many, that they may be saved.
\begin{flushright}I Corinthians 10:23-33\end{flushright}\end{quote}

Lest people forget what Paul said earlier, he reminds us that their is nothing wrong with the meat. I can in Christ eat any food I want, but there are some things that are not good to do and will harm me and others.  At one level idols are nothing, so there was no problem in getting cheap meat in the market (shambles was a meat market) from pagan temples or you could go to a friends house to enjoy a meal (feast), which has meat they had used in a ceremony.  That is like buying candy after Halloween because it is cheap then, which of course there is nothing wrong with.  There is nothing wrong with eating some candy from friends, even if it is Halloween left overs.  Paul reminds us though that we are not to be self centered, we are bought by Christ and we should care for our brothers and sisters.  If a brother or sister is troubled by me getting some cheap candy, because they think I am somehow participating in Halloween rather than just indulging my love of chocolate, then I should abstain.  Whatever I do I should do it for the sake of Christ, the edification of the saints, and reaching the lost.  Can you really say participating in Halloween does that?  Is God somehow honored by your participating in the universally acknowledged holiday of evil and terror?  It amazes me that I even have to write this refutation of Halloween participation.

\section{Answers to Objections}

Some may still protest that Samhain is not what they are celebrating.  They claim they are just having fun with costumes and getting tasty treats.  On one level this is true, most people are not aware of the history or significance of the holiday, but is that how someone who is trying to honor Jesus should appraise this?  You honestly must face what I Corinthians 8-10 is saying.  We are supposed to always behave in a manner which is honoring to God and to follow Christ.  If you are convinced that somehow ignorance is an excuse to God, or that maybe three consecutive chapters on one point are somehow being misunderstood, consider the following.
\begin{itemize}
\item The first several chapter's of Roman's indicates that God does not follow an appraisal that ignorance is an excuse for disobedience.  The ignorance excuse does not work with God.  Ignorance doesn't work with law enforcement, why try this on God?
\item I John 2 tells us that love of the world and its practice is hatred of God.  Jesus comments you can't serve two masters.  Why do you want to get away with sin anyway?  Do you love Jesus or the world?  This might sound harsh to some, but it is what the Bible says.
\item The letters to the seven churches of Revelations contains similar warnings to I Corinthians, such as Revelation 2:14.  Failing to head the warning includes a warning of Jesus fighting against the one who does not listen.
\end{itemize}
If you are going to follow Christ then follow Christ, if you want to follow the world then dress up and have fun.  Don't play the fence.

Others may claim I am being to legalistic or hard nosed on this.  It is true that many legalists object to Halloween, but that does not mean that is what I am or that my argument is wrong.  I know many legalists who condemn me, because I believe Christians are not under the law but under grace and that we have freedom to serve God.  My point is not for legalism, rather it is for pleasing God and growth in Christ.  Weigh what I have said.  Look at the Scriptures and history then honestly before God see what He wants you to do.   Legalism is all about rules you must follow to be holy, which is not my point at all.  I am saying don't practice another religion, even if it is unknowingly.  Celebrating Halloween is like participating in a Buddhist wedding ceremony, as both are participating in another religion.  The Old and New Testaments both condemn participating in other religion's worship.  If God says no, it is sinful to disobey.  Further it would not be pleasing to God for me to do something He said not to, so I would not be fulfilling my chief end.  Finally, we cannot hope to be conformed into the image of Christ if we constantly pursue our own desires and neglect His desires, thus growth is stunted.  In short, this is not a minor issue and this is not a legalistic point.

Still others say they have `Trunk or Treat' or a `harvest festival' instead of Halloween.  I have mentioned this briefly before, but people keep falling back to it.  I can sympathize with the desire, but let's be honest.  First, where you trick or treat is not the issue, the issue is this is a Celtic and Roman religious festival.  I don't care if you celebrate it traditionally or at your church, you are still celebrating it.  All that is being done is making the event safer and integrating false worship into your church's practice.  Second, Samhain is a harvest festival, so changing the language and name does not alter the event.  Dress it up, change the name, whatever, it does not change the fact this is another religion's holiday.  Finally, we all know these Halloween alternatives are just that, other ways to celebrate the holiday.  No one is suggesting having ``trunk or treat'' in August or a ``harvest festival'' in September.  Why?  Because churches are trying to celebrate Halloween without looking worldly.  In a word it is hypocrisy --- trying to look holy while sinning with everyone else.  If you are going to celebrate Halloween at least be honest enough to admit it.  More recent trends are to put it the day before and thus to expand the practice of Halloween to two days, so these efforts are actually making Halloween bigger and more popular and even including an explicit church approval.  If the idea ever was to lessen Halloween, then it has predictably backfired.

One final thing I hear over and over again is that my claim just ruins childhood and cuts their imagination.  Besides being a patently false and purely emotional argument that avoids dealing with the explicit statements of God, it directly harms children.  You might think I am crazy, but think carefully.  All Christians claim Heaven is the greatest, best, most fun, most nurturing place possible.  God created mankind, and knows best how to raise children correctly.  God says not to participate in pagan worship, and because kids like candy and dressing up we are going to say it is better not to listen to God?  Are we serious about that?  Either we don't believe God is right and Heaven is best, or we should reexamine our commitment to Halloween.  Can you imagine haunted houses, witches, vampires, demons, and pirates in Heaven?  If God does the best by kids and He does not do Halloween, of necessity we are doing bad by children by participating in Halloween.  My children are happy, creative, full of imagination, and none of them want to celebrate Halloween.  We have lots of fun without it, and they wonder, like I do, why anyone would want to worship Halloween.

\section{What I Do}

Quite frankly I do nothing, that is my whole point.  I am not advocating a different way to celebrate, I am advocating not celebrating at all.  I usually take my family out to a leisurely dinner, so I don't have to hear the doorbell while I eat.  When we are done we come home and spend the rest of the night as a family doing whatever we feel like.  We keep the front door light off, so people know we are not participating, and politely let those who ring anyway know we are not celebrating Halloween.  After a year or so, people get the point, and we have not had a doorbell rung in years.

I have occasionally had Bible studies on the truths of the reformation.  Years ago when we lived in Camarillo, my wife and I went around and stuck copies of the 95 theses on the elder's doors, as Martin Luther nailed the 95 theses on the church door in Wittenberg on October 31, 1517.  The point was misunderstood, and so even though we had fun we decided not to continue that one.  I try hard to not let these become a Halloween alternative, as I think the reformation should be celebrated anyway --- just not with costumes, candy, and jack-o-lanterns.  Whatever you do please don't let it become just an alternative.  If you are celebrating Reformation Day to avoid celebrating Halloween, please celebrate nothing.  If you could care less about Halloween and trying to provide a fun way to celebrate the great things God has done in preserving the gospel, then please celebrate Reformation Day.
