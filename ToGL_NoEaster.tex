\chapter{Easter Isn't Christian}

Why won't I call Resurrection Sunday, Easter Sunday?  Why won't I allow my kids to participate in Easter Egg hunts?  Why don't I want bunnies and colored eggs as part of the day?  Simply put, Easter is not a Christian holiday.

\section{History of Easter}
Easter (Eostre, Ostara) is the German/Saxon equivalent of the Babylonian goddess Astarte (Asherah in Canaan, Aphrodite in Greek).  She was the fertility/love goddess (also goddess of fortune, spring, and sometimes war) and was worshiped in many ways including fornicating with temple prostitutes and sacrificing rabbits (her symbol since they are so fertile).  Since she is the fertility goddess, the female hormone, estrogen, is named after her.  Her worshipers would also give pastel\footnote{Her colors were pastel as they were thought the colors of spring.} colored eggs on her high day (Easter, the week after vernal equinox- i.e. spring equinox).  As fertility goddess, she supposedly ushered in spring, hence why it was so special to them.  According to one legend about her, she found a bird whose wings were frozen and couldn't fly.  Easter turned the bird into a rabbit, but left the ability to lay eggs one day a year on the day that bears her name, Easter.  Her worshipers met in special groves and often worshiped her with the rising sun (her name also gives us the word east, where the sun rises).  The traditions of Easter are all her worship.  The Easter bunny was her pet (lupus was his name) as recounted above.  Egg hunts came about when the tradition of giving her eggs was condemned by the church, but people wanted to give them anyway.  They got around it by hiding them for others to find.  Sadly, Christians call Christ's resurrection day by the name of a pagan goddess and celebrate with the old pagan traditions.  Don't take my word for it, look it up in any reputable history of folktales, holidays, legends, or pagan beliefs.  Even do a search on the internet and you will get thousands of sources confirming this.

%Larry Boemler "Asherah and Easter," Biblical Archaeology Review, Vol. 18, Number 3, 1992-May/June
%Venerable Bede, (672-735 CE.), De Ratione Temporum - Easter named after Eostre

\section{Requirements of God}
How does God feel about this?  What does His word say?  What should we do?

The Old Testament has numerous references to the groves (literally ``asherah'' in the Hebrew), which were used in Asherah (remember Asherah is Easter) worship (1 Kings 15:13, 16:33, 2 Kings 13:6, 17:16, 21:3-7, 23:1-15, 2 Chronicles 15:16).  Notice God is never happy about them.  He always commands them to be destroyed.  God hates the worship of Asherah.  Asherah was consort of Baal, and their worship was often done together (see for example Judges 6:25-32).  In the Babylonian religion, Astarte (the Babylonians name for her) was the lover of Tammuz, who died each fall and rose each spring (as the god of nature).  On Astarte's high day (Easter) they would worship her by crying for Tammuz, the tears symbolizing the water that was supposed to bring him back to life.  God speaks against this practice in Ezekiel 8:13-18.  Notice the worship of this goddess is spoken against all over the Old Testament.  God hates it.

How can Christians participate in the worship of a false God, particularly one so condemned in Scripture?

I am sure many are thinking it is harmless and that I am over-reacting\footnote{I certainly admit I can be wrong on things, but I am not presenting my opinion, I have shown in scripture that God hates the celebration.  You must answer God, not me.}.  People don't think of it as Asherah worship anymore, you might be thinking.  You might even be wondering about Paul's comments about not worrying about eating meat sacrificed to idols in 1 Corinthians 8.  Paul was commenting that since there is no other god, Christians did not have to worry about if the meat came from a pagan temple.  Meat is meat.  His argument is not applicable here, because it is not either a cheap way to get necessities or an unknown origin of the necessities that are in view.  We are talking about participating in pagan practices, not eating food a pagan made.  If a non-christian friend invites you over to eat a meal his argument applies, but not in participating in pagan worships, that is covered in 1 Corinthians 10.  Paul's argument goes like this:
\begin{enumerate}
\item All the Israelites had all the benefits of participating in Old Testament worship, and the miracles of God, but God let them die in the wilderness. (v. 1-5)
\item This is an example for us as they participated in idolatry and God rejected them. Don't tempt Christ to reject you like they did.  Don't complain that you can't do something you think is fun.  Would you rather have that or Christ?  (v. 6-11)
\item If you think you are doing fine, be careful.  God does not allow temptation to be to great, and he leaves escapes so flee from the idolatry that tempts you. (v. 12-14)
\item How can you participate in Christian worship and pagan worship?  While there is no other god, there are devils and you cannot partake of their practices, it angers God.  (v. 15-22)
\item As a Christian we can enjoy the things God made but we need to make sure that we are not hurting others.  For instance in chapter 8 you were told not to worry about where the meat came from, but if a weaker brother knows and is bothered then don't eat it. (v. 23-30)
\item Do everything to God's glory.  Take every opportunity to share Christ based on your following the above.  (v. 31-33)
\end{enumerate}
The whole chapter is against pagan worship.  There is no room for ``it is harmless'' or ``everyone is doing it'' or even ``no one remembers its origins''.  You know the truth now.  God said to flee idolatry.  God said you can't follow pagan traditions and his.  God said liberty does not mean anything goes.  God said do all things to his glory.  How can you participate in a pagan ceremony to His glory?  Realize that the exhortation to not be unequally yoked in 2 Corinthians 6 was not only about marriage, it is about any participation in the sinful practices of the world.  Read Revelation 2, two churches are condemned because they permitted believers to partake of pagan practices (the reference to eating meat sacrificed to idols is not just eating the food but eating it at a pagan feast).  If you didn't like the candy, foods, and traditions would you try to defend them?  In other words are you really sure God wants you to participate in pagan activities He condemned, or do you just want to have your fun?  Be honest, you can't hide the truth from Him.  God commands us to obey Him not our own desires.


John 14:15, 21, 15:10, 14-15, 1 John 5:1-5, 2 John 4-6
